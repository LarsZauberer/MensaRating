\chapter{Resultate und Testen} \label{chap:resultate}

\section{Resultate} \label{sec:resultate}

Als Resultat dieser Arbeit ist eine voll funktionsfähige Webseite entstanden.
Die Webseite erfüllt alle nötigen Anforderungen (siehe
\ref{sec:problemdefinition}). Die Webseite ist auch öffentlich zugänglich
(siehe \ref{spez:Deployment}) und kann unter
\url{http://mensarating.herokuapp.com/} aufgerufen werden. Es lässt sich also
sagen, dass die Webseite komplett einsatzbereit ist und Nutzer von der NKSA sie
beginnen können zu benutzen.

\section{Testen} \label{sec:testen}

Die Webseite lässt sich testen in einer lokalen Entwicklungsumgebung oder auf
der öffentlichen Seite. Getestet wurden die Funktionen manuell von Hand oder
durch automatisierte Tests.

Grundsätzlich funktionieren alle Funktionen und man kann sie auch nutzen.
Insgesamt gibt es nur ein Problem. Es kann sein, dass der Webscraper manchmal in
Edge-Cases hineinläuft. Das kann passieren, wenn die offizielle Mensaseite
ihr Menu zu schlechten Zeiten ändert oder allgemein wenn die Mensa Informationen
auf der Seite ändert. Es wurde versucht diese Edge-Cases möglichst alle
abzufangen, allerdings kann es immer noch sein, dass es manchmal zu Fehlern
kommt. Diese Fehler müssten dann von einem Admin auf der Admin-Seite behoben
werden.
