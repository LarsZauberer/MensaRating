\chapter{Spezifikationen}
\section{Datenbank}\label{spez:DB} Die Datenbank muss Daten speichern können. Es
wird eine PostGreSQL Datenbank verwendet (siehe \ref{fig:Website}). Die
Steuerung läuft über \code{Django} in der \code{models.py} Datei (siehe
\ref{code:core.models.py}). Die Migrierung der Datenbank wird von \code{Django}
in den \code{Migrationfiles} gemacht.

Eine anschauliche Darstellung der Datenbank ist in \ref{fig:DB} zu sehen.

\subsection{Entities}
\subsubsection{Profil}\label{model:Profil}
Da \code{Django} es nicht zulässt, dem \code{User}-Model weitere Attribute
hinzufügen, braucht es eine eigene Entität \code{Profil}. Diese wird mit einem
\code{OneToOneField} mit dem \code{User} verknüpft. Weiter enthält es ein Karma
Attribut um Punkte für das Punktesystem zu speichern.

\subsubsection{Menu}\label{model:Menu} Das \code{Menu} Model speichert die Menus
der Mensa. Jeder Menueintrag entspricht einem einzelnen vorkommen dieses
Gerichtes. Wenn also ein Gericht 2 Mal vorkommt, gibt es auch 2 Menu einträge.
Aus diesem Grund werden die \code{Menu}-Objekte mit einem \code{ForeignKey} auf
das \code{MenuType} verknüpft (siehe \ref{model:MenuType}). Im Code dient als
Zuordnungs Attribut für die \code{MenuTypes} der \code{Name} des Menus. Insgesamt enthält es
folgende Attribute:
\begin{itemize}
    \item \code{Name}
    \item \code{Description}
    \item \code{Date} (Datum, wann das Menu verfügbar war)
    \item \code{Vegetarian}
    \item \code{Vegan}
    \item \code{MenuType} (siehe \ref{model:MenuType})
\end{itemize}

\subsubsection{MenuType}\label{model:MenuType} Das \code{MenuType} ist ein
Model, welches die verschiedenen Vorkommen der \code{Menu}-Objekte (siehe
\ref{model:Menu}) dynamisch zuordnet. Es hat nur einen \code{Namen} als Attribut. Das
wird als Zuordnungs Attribut für die Menus verwendet.

\subsubsection{Review}\label{model:Review} Das \code{Review} Model speichert die
Text Bewertungen zu den Menus. Es enthält eine verknüpfung zu dem \code{Profil}
(siehe \ref{model:Profil}). Wenn der Nutzer keinen Account und somit kein Profil
hat (siehe \ref{spez:AccountSystem}) ist das Attribut \code{null}. Es enhält
auch eine Referenz auf das \code{Menu} (siehe \ref{model:Menu}). Weiter
speichert es seine \code{likes} den \code{Bewertungstext} und das \code{Erstellungsdatum}.

\subsubsection{Image}\label{model:Image} Das \code{Image} Model enthält
dieselben Attribute wie ein Review (siehe \ref{model:Review}). Allerdings hat es
natürlich keinen Bewertungstext, sondern die Referenz auf den Speicherort eines
Bildes. Je nach Server ist der Speicherort lokal (Development Environment) oder
auf einem Google Drive (deployed auf Heroku).

\subsubsection{Rating}\label{model:Rating} Das \code{Rating} Model speichert das
\code{Profil}, welches das Rating hinterlässt. Wenn der Nutzer keinen Account
und somit kein Profil hat (siehe \ref{spez:AccountSystem}), ist das Attribut
\code{null}. Es enhält auch eine Referenz auf das \code{Menu} (siehe
\ref{model:Menu}). Das Attribut \code{Rating} ist dann eine Zahl zwischen
\textbf{1 und 5}.

\subsubsection{Badge}\label{model:Badge} Das \code{Badge} Model gehört zu dem
Achievement System (siehe \ref{spez:Badges}). Es enthält einen \code{Namen},
eine \code{Beschreibung} und ein \code{Bild}. Weiter hat es 2 Attribute:
\code{condition\_category} und \code{count}, welche im Achievement System genauer
erklärt werden (siehe \ref{spez:Badges}).

\section{index page}
\subsection{index view}
\subsection{index template}

\section{menu page}
\subsection{menu view}
\subsubsection{menu view get}
\subsubsection{menu view post}
\subsection{menu template}

\section{Posting Forms}

\section{menuType page}
\subsection{menuType view}
\subsection{menuType template}

\section{allMenu page}
\subsection{allMenu view}
\subsection{allMenu template}

\section{timeline page}
\subsection{timeline view}
\subsection{timeline template}

\section{userprofile page}
\subsection{userprofile view}
\subsection{userprofile template}

\section{like request}

\section{Account System} \label{spez:AccountSystem}
\subsection{register page}
\subsubsection{register view}
\subsubsection{register template}

\subsection{login page}
\subsubsection{login view}
\subsubsection{login template}

\subsection{logout page}
\subsubsection{logout view}
\subsubsection{logout template}

\subsection{password reset}

\section{Webscraper}

\section{Achievements (Badges)} \label{spez:Badges}
\section{Punktesystem (Karma)}
