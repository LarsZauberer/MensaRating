\chapter{Spezifikationen}

\section{Menus} \label{spez:Menus}

Menus sind Objekte, welche in der Datenbank als \code{Menu}-Objekt gespeichert
werden (siehe \ref{fig:DB}, \ref{code:core.models.py}). Jedes \code{Menu} ist
ein Vorkommen eines Gerichts. Um die verschiedenen Vorkommen der
\code{Menu}-Objekte zu gruppieren existiert das \code{MenuType}-Objekt. Als
Gruppierungsattribut dient der Name des Gerichts.

Die Menus werden von der Mensa-Website gescraped. Die Synchronization (siehe
\ref{spez:Webscraper}) findet bei jedem Aufruf einer der Seiten statt.

Nur wenn das heutige Datum mit dem Datum des Menus übereinstimmt, ist es möglich
Dinge zu posten (siehe \ref{spez:Posts}), das Menu zu Bewerten (siehe
\ref{spez:Rating}) und die Posts zu liken (siehe \ref{spez:Liking}).

\section{Webscraper} \label{spez:Webscraper}

Der Webscraper ist ein standalone Python Script (siehe
\ref{code:core.webscraper.py}). Der Webscraper stellt mit der Library
\code{requests} Anfragen an die Webseite
\url{https://neuekanti.sv-restaurant.ch/de/menuplan/}. Zuerst werden die Tages-
und Datumsdaten von der Seite geladen. Danach werden die Gerichte (Name,
Beschreibung, Vegan/Vegetarisch) gescraped.

Das Script wird bei jedem Aufruf der Webseite ausgeführt. Nach dem Scraping
der Daten werden diese mit der Datenbank (siehe \ref{fig:DB}) verglichen.
Ist das \code{Menu}-Objekt (siehe \ref{spez:Menus}) noch nicht in der Datenbank,
so wird nach einem zugehörigen \code{MenuType} (siehe \ref{spez:Menus})
gesucht. Wenn dieses nicht existiert, dann werden beide Objekte einfach mit den
Daten erstellt. Sonst wird nur das \code{Menu}-Objekt erstellt.

\begin{lstlisting}
    data = scrape_data()
    menus = get_menus_from_db()
    for menu in data:
        if menu not in menus:
            menu_type = get_menu_type(menu)
            if menu_type is None:
                menu_type = create_menu_type(menu)
            create_menu(menu, menu_type)
\end{lstlisting}

\section{Bilder Gallerie} \label{spez:Gallerie}

Die Bildergallerie ist ein Frontend Feature. Die Bildergallerie wurde von dem
Tutorial auf dieser Seite nachgemacht:
\url{https://www.w3schools.com/howto/howto_js_slideshow.asp}. Die Bildergallerie
wird gebraucht, damit die Images (siehe \ref{spez:Images}) der Menus (siehe
\ref{spez:Menus}) angezeigt werden können.

Im Javascript werden die verschiedenen Bilder in einem Array gespeichert. Nur
das aktive Bild bekommt den style \code{display: block}. Die anderen Bilder
haben \code{display: none}.

Wenn es noch keine Bilder von einem Menu gibt, dann wird ein default Bild
angezeigt.

Da Bilder Hoch- oder Querformat sein können, werden sie auf ihre Orientierung
überprüft. Wenn das Bild Hochformat ist, dann wird es im css anders
behandelt.

\section{Posts} \label{spez:Posts}

Der Begriff "Posts" beschreibt zwei verschiedene Objekte in der Datenbank (siehe \ref{fig:DB}). Es gibt
kein übergeordnetes Objekt Posts. Als Posts werden Images (siehe \ref{spez:Images})
und Reviews (siehe \ref{spez:Reviews}) bezeichnet.

\subsection{Images} \label{spez:Images}

Es gibt ein \code{Image}-Objekt in der Datenbank (siehe \ref{fig:DB} und in
\ref{code:core.models.py}). Die Image Datei wird nicht in der Datenbank
gespeichert, lediglich eine Referenz, welche auf das Bild verweist.

Wenn das Image auf der öffentlichen Webseite (siehe \ref{spez:Deployment}) angezeigt wird, dann
wird das Image nicht auf dem eigentlichen Server gespeichert, sondern auf einer
Google Drive. Das wird gemacht, da der Server Container keinen persistant
Storage hat.

\subsection{Reviews} \label{spez:Reviews}

Es gibt ein \code{Review}-Objekt in der Datenbank (siehe \ref{fig:DB} und in
\ref{code:core.models.py}). Ein Review ist ein Kommentar, welcher auf der
Webseite angezeigt werden kann.

\subsection{Liking} \label{spez:Liking}

Man kann die Posts liken. Dabei wird der Like Counter in der Datenbank erhöht.
Ob der User bereits geliked hat, wird im Javascript Localstorage gespeichert.
Das wird gemacht, damit die Datenbank nicht zu viele Einträge bekommt.

\section{Rating} \label{spez:Rating}

Ein Rating ist ein Objekt in der Datenbank (siehe \ref{fig:DB} und in
\ref{code:core.models.py}). Ein Rating ist eine Bewertung für ein Menu-Objekt
(siehe \ref{spez:Menus}). Ein Rating hat einen Wert zwischen 1 und 5.

\section{Statistik Filter und Sortierung} \label{spez:Statistik}



\section{Account System} \label{spez:Account}

Django hat ein eingebautes Accountsystem. Dieses Accountsystem wird in die
Webseite integriert (siehe \ref{code:users.views.py}). Es bietet folgende Funktionen:
\begin{itemize}
    \item register
    \item login
    \item logout
    \item password-reset
    \item password-reset-done
    \item password-reset-confirm
    \item password-reset-complete
\end{itemize}

Beim Registrieren wird ein \code{User}-Objekt in der Datenbank (noch nicht in der
models.py Datei) erstellt. Danach auch ein \code{Profil}-Objekt (siehe
\ref{code:core.models.py} und \ref{fig:DB}). Beim Registrieren gibt es auch noch
eine ReCaptcha-Überprüfung, damit die Datenbank nicht zugespammt werden kann. Eine Email-Verifikation gibt es nicht

Login und Logout verwalten die Session mit den von Django implementierten
Funktionen.

Die Funktion des Passwordresets funktioniert über die angegebene Email Adresse.
Es wird ein sicherer Token für den Passwordreset an die Email Adresse gesendet.
Das Email Adresse wird mit einem Gmail Account gesendet.

\section{Mobile Responsiveness} \label{spez:Mobile}

In Milestone 1 noch nicht implementiert.

\section{Punktesystem (Karma)} \label{spez:Karma}

Die Erfahrungspunkte, auch Karma genannt, werden in \code{Profil}-Objekt (siehe
\ref{fig:DB} und \ref{code:core.models.py}) gespeichert.

Man kann Punkte für das Veröffentlichen von Posts (siehe \ref{spez:Posts}) und für
Likes auf den eigenen Posts bekommen.

Mit den Punkten kann man Achievements erreichen (siehe \ref{spez:Badges}).

\section{Achievements (Badges)} \label{spez:Badges}

Die Achievements werden als \code{Badge}-Objekt in der Datenbank (siehe \ref{fig:DB} und in
\ref{code:core.models.py}) gespeichert. Die Objekte werden keinem User
zugeordnet gespeichert, sondern dynamisch berechnet, wer welche Achievements
hat.

Es gibt 3 verschiedene Kategorien von Achievements.
\begin{itemize}
    \item Karma-basiert (siehe \ref{spez:Karma})
    \item Images-basiert (siehe \ref{spez:Images})
    \item Review-basiert (siehe \ref{spez:Reviews})
\end{itemize}

Wenn man ein besseres Achievements in einer Kategorie erreicht, dann wird das alte durch das neue ersetzt. Man kann Achievements auch wieder verlieren, wenn man die Kriterien nicht mehr erfüllt.

Karma-basiert heisst, man bekommt ein Achievment in dieser Kategorie, wenn man
ein eine bestimmte Menge an Karma überschritten hat.

Image- und Review-basiert heisst, dass man eine bestimmte Anzahl an "most liked
Images/Reviews" in einem MenuType (siehe \ref{spez:Menus}) hat.

Alle erreichten Achievements sind im Profil ersichtlich (siehe \ref{code:core.views.py})

\section{Deployment} \label{spez:Deployment}

Die Webseite wird auf einem Heroku Eco Dyno gehostet. Die Datenbank wird ebenfalls
von Heroku auf einem Postgres Server gehostet. Die Bilder werden auf einer
Google Drive gespeichert (siehe \ref{spez:Images}).
