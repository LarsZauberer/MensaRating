\chapter{Spezifikationen}

\section{Menus} \label{spez:Menus}

Menus sind Objekte, welche in der Datenbank als \code{Menu}-Objekt gespeichert
werden (siehe \ref{fig:DB}, \ref{code:core.models.py}). Jedes \code{Menu} ist
ein Vorkommen eines Gerichts. Um die verschiedenen Vorkommen der
\code{Menu}-Objekte zu gruppieren existiert das \code{MenuType}-Objekt. Als
Gruppierungsattribut dient der Name des Gerichts.

Die Menus werden von der Mensa-Website gescraped. Die Synchronization (siehe
\ref{spez:Webscraper}) findet bei jedem Aufruf einer der Seiten statt.

\section{Webscraper} \label{spez:Webscraper}

Der Webscraper ist ein standalone Python Script (siehe
\ref{code:core.webscraper.py}). Der Webscraper stellt mit der Library
\code{requests} Anfragen an die Webseite
\url{https://neuekanti.sv-restaurant.ch/de/menuplan/}. Zuerst werden die Tages
und Datumsdaten von der Seite geladen. Danach werden die Gerichte (Name,
Beschreibung, Vegan/Vegetarisch) gescraped.

Das Script wird bei jedem Aufruf von der Webseite ausgeführt. Nach dem Scraping
der Daten werden die Daten mit der Datenbank (siehe \ref{fig:DB}) verglichen.
Ist das \code{Menu}-Objekt (siehe \ref{spez:Menus}) noch nicht in der Datenbank,
dann wird nach einem zugehörigen \code{MenuType} (siehe \ref{spez:Menus})
gesucht. Wenn dieses nicht existiert, dann werden beide Objekte einfach mit den
Daten erstellt. Sonst wird nur das \code{Menu}-Objekt erstellt.

\begin{lstlisting}
    data = scrape_data()
    menus = get_menus_from_db()
    for menu in data:
        if menu not in menus:
            menu_type = get_menu_type(menu)
            if menu_type is None:
                menu_type = create_menu_type(menu)
            create_menu(menu, menu_type)
\end{lstlisting}

\section{Bilder Gallerie} \label{spez:Gallerie}
\section{Posts} \label{spez:Posts}
\subsection{Images} \label{spez:Images}
\subsection{Reviews} \label{spez:Reviews}
\section{Rating} \label{spez:Rating}
\section{Statistik Filter und Sortierung} \label{spez:Statistik}
\section{Account System} \label{spez:Account}
\section{Mobile Responsiveness} \label{spez:Mobile}
\section{Punktesystem (Karma)} \label{spez:Karma}
\section{Achievements (Badges)} \label{spez:Badges}
\section{Deployment} \label{spez:Deployment}
